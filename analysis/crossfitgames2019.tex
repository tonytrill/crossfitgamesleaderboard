\documentclass[]{article}
\usepackage{lmodern}
\usepackage{amssymb,amsmath}
\usepackage{ifxetex,ifluatex}
\usepackage{fixltx2e} % provides \textsubscript
\ifnum 0\ifxetex 1\fi\ifluatex 1\fi=0 % if pdftex
  \usepackage[T1]{fontenc}
  \usepackage[utf8]{inputenc}
\else % if luatex or xelatex
  \ifxetex
    \usepackage{mathspec}
  \else
    \usepackage{fontspec}
  \fi
  \defaultfontfeatures{Ligatures=TeX,Scale=MatchLowercase}
\fi
% use upquote if available, for straight quotes in verbatim environments
\IfFileExists{upquote.sty}{\usepackage{upquote}}{}
% use microtype if available
\IfFileExists{microtype.sty}{%
\usepackage{microtype}
\UseMicrotypeSet[protrusion]{basicmath} % disable protrusion for tt fonts
}{}
\usepackage[margin=1in]{geometry}
\usepackage{hyperref}
\hypersetup{unicode=true,
            pdftitle={2019 CrossFit Open, The Real Numbers Behind Open Participation},
            pdfborder={0 0 0},
            breaklinks=true}
\urlstyle{same}  % don't use monospace font for urls
\usepackage{color}
\usepackage{fancyvrb}
\newcommand{\VerbBar}{|}
\newcommand{\VERB}{\Verb[commandchars=\\\{\}]}
\DefineVerbatimEnvironment{Highlighting}{Verbatim}{commandchars=\\\{\}}
% Add ',fontsize=\small' for more characters per line
\usepackage{framed}
\definecolor{shadecolor}{RGB}{248,248,248}
\newenvironment{Shaded}{\begin{snugshade}}{\end{snugshade}}
\newcommand{\KeywordTok}[1]{\textcolor[rgb]{0.13,0.29,0.53}{\textbf{#1}}}
\newcommand{\DataTypeTok}[1]{\textcolor[rgb]{0.13,0.29,0.53}{#1}}
\newcommand{\DecValTok}[1]{\textcolor[rgb]{0.00,0.00,0.81}{#1}}
\newcommand{\BaseNTok}[1]{\textcolor[rgb]{0.00,0.00,0.81}{#1}}
\newcommand{\FloatTok}[1]{\textcolor[rgb]{0.00,0.00,0.81}{#1}}
\newcommand{\ConstantTok}[1]{\textcolor[rgb]{0.00,0.00,0.00}{#1}}
\newcommand{\CharTok}[1]{\textcolor[rgb]{0.31,0.60,0.02}{#1}}
\newcommand{\SpecialCharTok}[1]{\textcolor[rgb]{0.00,0.00,0.00}{#1}}
\newcommand{\StringTok}[1]{\textcolor[rgb]{0.31,0.60,0.02}{#1}}
\newcommand{\VerbatimStringTok}[1]{\textcolor[rgb]{0.31,0.60,0.02}{#1}}
\newcommand{\SpecialStringTok}[1]{\textcolor[rgb]{0.31,0.60,0.02}{#1}}
\newcommand{\ImportTok}[1]{#1}
\newcommand{\CommentTok}[1]{\textcolor[rgb]{0.56,0.35,0.01}{\textit{#1}}}
\newcommand{\DocumentationTok}[1]{\textcolor[rgb]{0.56,0.35,0.01}{\textbf{\textit{#1}}}}
\newcommand{\AnnotationTok}[1]{\textcolor[rgb]{0.56,0.35,0.01}{\textbf{\textit{#1}}}}
\newcommand{\CommentVarTok}[1]{\textcolor[rgb]{0.56,0.35,0.01}{\textbf{\textit{#1}}}}
\newcommand{\OtherTok}[1]{\textcolor[rgb]{0.56,0.35,0.01}{#1}}
\newcommand{\FunctionTok}[1]{\textcolor[rgb]{0.00,0.00,0.00}{#1}}
\newcommand{\VariableTok}[1]{\textcolor[rgb]{0.00,0.00,0.00}{#1}}
\newcommand{\ControlFlowTok}[1]{\textcolor[rgb]{0.13,0.29,0.53}{\textbf{#1}}}
\newcommand{\OperatorTok}[1]{\textcolor[rgb]{0.81,0.36,0.00}{\textbf{#1}}}
\newcommand{\BuiltInTok}[1]{#1}
\newcommand{\ExtensionTok}[1]{#1}
\newcommand{\PreprocessorTok}[1]{\textcolor[rgb]{0.56,0.35,0.01}{\textit{#1}}}
\newcommand{\AttributeTok}[1]{\textcolor[rgb]{0.77,0.63,0.00}{#1}}
\newcommand{\RegionMarkerTok}[1]{#1}
\newcommand{\InformationTok}[1]{\textcolor[rgb]{0.56,0.35,0.01}{\textbf{\textit{#1}}}}
\newcommand{\WarningTok}[1]{\textcolor[rgb]{0.56,0.35,0.01}{\textbf{\textit{#1}}}}
\newcommand{\AlertTok}[1]{\textcolor[rgb]{0.94,0.16,0.16}{#1}}
\newcommand{\ErrorTok}[1]{\textcolor[rgb]{0.64,0.00,0.00}{\textbf{#1}}}
\newcommand{\NormalTok}[1]{#1}
\usepackage{graphicx,grffile}
\makeatletter
\def\maxwidth{\ifdim\Gin@nat@width>\linewidth\linewidth\else\Gin@nat@width\fi}
\def\maxheight{\ifdim\Gin@nat@height>\textheight\textheight\else\Gin@nat@height\fi}
\makeatother
% Scale images if necessary, so that they will not overflow the page
% margins by default, and it is still possible to overwrite the defaults
% using explicit options in \includegraphics[width, height, ...]{}
\setkeys{Gin}{width=\maxwidth,height=\maxheight,keepaspectratio}
\IfFileExists{parskip.sty}{%
\usepackage{parskip}
}{% else
\setlength{\parindent}{0pt}
\setlength{\parskip}{6pt plus 2pt minus 1pt}
}
\setlength{\emergencystretch}{3em}  % prevent overfull lines
\providecommand{\tightlist}{%
  \setlength{\itemsep}{0pt}\setlength{\parskip}{0pt}}
\setcounter{secnumdepth}{0}
% Redefines (sub)paragraphs to behave more like sections
\ifx\paragraph\undefined\else
\let\oldparagraph\paragraph
\renewcommand{\paragraph}[1]{\oldparagraph{#1}\mbox{}}
\fi
\ifx\subparagraph\undefined\else
\let\oldsubparagraph\subparagraph
\renewcommand{\subparagraph}[1]{\oldsubparagraph{#1}\mbox{}}
\fi

%%% Use protect on footnotes to avoid problems with footnotes in titles
\let\rmarkdownfootnote\footnote%
\def\footnote{\protect\rmarkdownfootnote}

%%% Change title format to be more compact
\usepackage{titling}

% Create subtitle command for use in maketitle
\newcommand{\subtitle}[1]{
  \posttitle{
    \begin{center}\large#1\end{center}
    }
}

\setlength{\droptitle}{-2em}

  \title{2019 CrossFit Open, The Real Numbers Behind Open Participation}
    \pretitle{\vspace{\droptitle}\centering\huge}
  \posttitle{\par}
    \author{}
    \preauthor{}\postauthor{}
    \date{}
    \predate{}\postdate{}
  

\begin{document}
\maketitle

\subsection{Introduction}\label{introduction}

My name is Tony Silva. I am an above average CrossFitter that really
enjoys looking at data and numbers. As the self proclaimed Chief Data
Scientist for Koda CrossFit Iron View, I took it upon myself to build
some programming scripts to pull all of the data from the CrossFit Games
Open web API. If you don't know what that means, I basically built
something to save all the data from the Open leaderboard that you can
access from \textless{}games.crossfit.com\textgreater{}. I have been
able to capture, last year's data and this year's data. I built some
scripts that pre-process the data into usable formats that can be
analyzed. I used that data to build out the analysis in this blog on
finding the true numbers behind the CrossFit Open participation.

There has been a lot of change to the CrossFit Games Season this year,
beginning with the Open. As many of you might know the Regionals were
taken out and now you can make it to the CrossFit Games directly from
the Open. You can be number one in your country at the end of the 5
weeks, or you can be in the top 20 of the Open to make it. Because of
the change in the Games season, many people believe that sign ups for
the Open will be down since last year's. Early reports, from Morning
Chalk Up, show that sign ups are down 32\% found here:
\url{https://morningchalkup.com/2019/02/22/correction-open-registration-down-only-32-earlier-reports-inaccurate/}.
While these numbers are close to true, they aren't the actuals when
pulling directly from the leaderboard.

\paragraph{How many people have signed up in 2018
vs.~2019?}\label{how-many-people-have-signed-up-in-2018-vs.2019}

I was able to use R, a data analysis tool, to build some code that
counts the number of athletes from each Open season and coompares them
together. Using my the code below, you can see that were roughly 399489
sign ups in 2018.

\begin{Shaded}
\begin{Highlighting}[]
\NormalTok{num_athletes_}\DecValTok{18}\NormalTok{ <-}\StringTok{ }\KeywordTok{length}\NormalTok{(}\KeywordTok{unique}\NormalTok{(athletes_}\DecValTok{2018}\OperatorTok{$}\NormalTok{competitorId))}
\NormalTok{num_athletes_}\DecValTok{18}
\end{Highlighting}
\end{Shaded}

\begin{verbatim}
## [1] 399489
\end{verbatim}

And below you can see the number of sign ups in 2019.

\begin{Shaded}
\begin{Highlighting}[]
\NormalTok{num_athletes_}\DecValTok{19}\NormalTok{ <-}\StringTok{ }\KeywordTok{length}\NormalTok{(}\KeywordTok{unique}\NormalTok{(athletes_}\DecValTok{2019}\OperatorTok{$}\NormalTok{competitorId))}
\NormalTok{num_athletes_}\DecValTok{19}
\end{Highlighting}
\end{Shaded}

\begin{verbatim}
## [1] 226500
\end{verbatim}

We can actually can see the percentage change in Open participation from
2018 to 2019.

\begin{Shaded}
\begin{Highlighting}[]
\DecValTok{1}\OperatorTok{-}\KeywordTok{round}\NormalTok{(num_athletes_}\DecValTok{19}\OperatorTok{/}\NormalTok{num_athletes_}\DecValTok{18}\NormalTok{,}\DecValTok{4}\NormalTok{)}
\end{Highlighting}
\end{Shaded}

\begin{verbatim}
## [1] 0.433
\end{verbatim}

43\% is signficantly different than the 32\% being reported by the
Morning Chalk Up and other media outlets. I actually think the original
number they posted of 44\% is a more accruate number. Either way, these
numbers don't look great for CrossFit HQ and the Open.

While the total number of athletesand the change in total athletes is
important, what's really important is to identify the number of new
athletes that have signed up this year vs.~the number of athletes who
signed up both 2018 and 2019. Both are interesting metrics to monitor
for the ``health'' of the Open. To do that, I joined the 2018 data and
2019 data using the ``CompetitorId'' field that is housed in the API. By
doing this, we can find the IDs that exist in the 2019 data, and not the
2018 data (New Athletes), IDs that exist in 2018 and 2019 (Repeat
Athletes), and IDs that exist in 2018 but not 2019 (Athletes Who Didn't
Sign Up in 2019). You can see this from the code below:

\begin{Shaded}
\begin{Highlighting}[]
\CommentTok{# Create a athletes dataframe}
\NormalTok{athletes <-}\StringTok{ }\NormalTok{athletes_}\DecValTok{2019} \OperatorTok
\StringTok{  }\KeywordTok{left_join}\NormalTok{(athletes_}\DecValTok{2018}\NormalTok{, }\DataTypeTok{by=}\StringTok{'competitorId'}\NormalTok{)  }\OperatorTok
\StringTok{  }\KeywordTok{mutate}\NormalTok{(}\DataTypeTok{IsNew=}\KeywordTok{ifelse}\NormalTok{(}\KeywordTok{is.na}\NormalTok{(competitorName.y),}\DecValTok{1}\NormalTok{,}\DecValTok{0}\NormalTok{))}
\end{Highlighting}
\end{Shaded}

Below is the number of new Athletes that signed up.

\begin{Shaded}
\begin{Highlighting}[]
\CommentTok{# New Athletes}
\KeywordTok{sum}\NormalTok{(athletes}\OperatorTok{$}\NormalTok{IsNew)}
\end{Highlighting}
\end{Shaded}

\begin{verbatim}
## [1] 103575
\end{verbatim}

Below is the number of Repeat Athletes that signed up.

\begin{Shaded}
\begin{Highlighting}[]
\CommentTok{# Repeat Atheltes}
\NormalTok{num_athletes_}\DecValTok{19} \OperatorTok{-}\StringTok{ }\KeywordTok{sum}\NormalTok{(athletes}\OperatorTok{$}\NormalTok{IsNew)}
\end{Highlighting}
\end{Shaded}

\begin{verbatim}
## [1] 122925
\end{verbatim}

We can then take the repeat athletes and determine, of those what's the
percentage of people who signed up last year that also signed up this
year?

\begin{Shaded}
\begin{Highlighting}[]
\KeywordTok{round}\NormalTok{((num_athletes_}\DecValTok{19} \OperatorTok{-}\StringTok{ }\KeywordTok{sum}\NormalTok{(athletes}\OperatorTok{$}\NormalTok{IsNew))}\OperatorTok{/}\NormalTok{num_athletes_}\DecValTok{18}\NormalTok{,}\DecValTok{4}\NormalTok{)}\OperatorTok{*}\DecValTok{100}
\end{Highlighting}
\end{Shaded}

\begin{verbatim}
## [1] 30.77
\end{verbatim}

So roughly XX\% of the people who signed up last year have signed up
this year. I believe this number is much more drastic to what the
Morning Chalk Up has been writing about. Morning Chalk Up is comparing
the numbers of total number of participants in both years. When in my
opinion the health of the Open and the sport of CrossFit depends on
CrossFit's ability to retain participants. So what does this mean? Can
we say that the change to the Open caused a statistically significant
change in signups from existing CrossFitters? Since I am only looking at
two years of data, it's hard to say. But the numbers are pretty drastic.

\begin{Shaded}
\begin{Highlighting}[]
\NormalTok{athletes_}\DecValTok{2019} \OperatorTok
\StringTok{  }\KeywordTok{group_by}\NormalTok{(countryOfOriginName) }\OperatorTok
\StringTok{  }\KeywordTok{summarise}\NormalTok{(}\DataTypeTok{athletes =} \KeywordTok{n}\NormalTok{()) }\OperatorTok
\StringTok{  }\KeywordTok{arrange}\NormalTok{(}\KeywordTok{desc}\NormalTok{(athletes))}
\end{Highlighting}
\end{Shaded}

\begin{verbatim}
## # A tibble: 130 x 2
##    countryOfOriginName athletes
##    <chr>                  <int>
##  1 United States         112937
##  2 United Kingdom         12351
##  3 Brazil                 12340
##  4 Canada                 12053
##  5 Australia              11802
##  6 France                 11318
##  7 Germany                 4138
##  8 Spain                   3555
##  9 Netherlands             3368
## 10 Italy                   3147
## # ... with 120 more rows
\end{verbatim}

\paragraph{Is CrossFit Actually Increasing International
Participation?}\label{is-crossfit-actually-increasing-international-participation}

We can see from the numbers that nearly half of all sign ups are new
people. CrossFit has been trying hard to push International
participation in the Open in order to get the sport more wide spread. We
can dive into the data and see how the share of international
participants has changed year over year. We can see the percent share of
international athletes from last year below:

\begin{Shaded}
\begin{Highlighting}[]
\KeywordTok{round}\NormalTok{(}\KeywordTok{sum}\NormalTok{(athletes_}\DecValTok{2018}\OperatorTok{$}\NormalTok{IsInternational)}\OperatorTok{/}\NormalTok{num_athletes_}\DecValTok{18}\NormalTok{,}\DecValTok{4}\NormalTok{)}
\end{Highlighting}
\end{Shaded}

\begin{verbatim}
## [1] 0.4538
\end{verbatim}

And we can see the percent share of international athletes this year.

\begin{Shaded}
\begin{Highlighting}[]
\KeywordTok{round}\NormalTok{(}\KeywordTok{sum}\NormalTok{(athletes_}\DecValTok{2019}\OperatorTok{$}\NormalTok{IsInternational)}\OperatorTok{/}\NormalTok{num_athletes_}\DecValTok{19}\NormalTok{,}\DecValTok{4}\NormalTok{)}
\end{Highlighting}
\end{Shaded}

\begin{verbatim}
## [1] 0.5006
\end{verbatim}

We can see that the share of international athletes partcipation is
bigger than last year. However, this could be due to a decrease in US
participation and not necessarily a


\end{document}
